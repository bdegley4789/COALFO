% ===============================================
% CS 461: Requirements         Fall 2017
% hw_revised.tex
% Template for revised homework submission
% ===============================================
%         READ THE FOLLOWING CAREFULLY!!!
% ===============================================
% When you produce a PDF version of this document
% to turn in, change the filename to hwX-name.pdf
% replacing X with the homework assignment number
% and name with your last name.
% ===============================================

% -----------------------------------------------
% The preamble that follows can be ignored. Go on
% down to the section that says "START HERE"
% -----------------------------------------------
\documentclass[a4paper,12pt]{article}
\usepackage[colorlinks = true,
            linkcolor = blue,
            urlcolor  = blue,
            citecolor = blue,
            anchorcolor = blue]{hyperref}
\begin{document}
\sloppy

\noindent CS 461 Requirements\newline


\noindent Team Name: \href{https://capstone-coal.github.io/team}{COAL-FO}\newline


\noindent Bryce Egley, Kenny Thompson\newline


\noindent {\fontsize{14pt}{14pt}\selectfont \textbf{1. Introduction}}\newline


\noindent \textbf{1.1 Purpose}\newline


\noindent The purpose for this document is to layout the requirements for the \href{https://capstone-coal.github.io/team}{COAL-FO} (COAL Follow On) capstone project and describe how these requirements will be accomplished. \newline

\noindent The purpose of our \href{https://capstone-coal.github.io/team}{COAL-FO} project is to improve pycoal algorithms, enable more spectral libraries to work with pycoal, to collect more data to process with pycoal and to improve general functionality of pycoal. \newline

\noindent \textbf{1.2 Scope}\newline


\noindent The project will be focused around publicly available \href{https://capstone-coal.github.io/}{COAL} algorithms created by the previous capstone group, publicly available spectral analysis collected from the NASA \href{https://aviris.jpl.nasa.gov/}{AVIRIS} project. Work for this project will be stored in the \href{https://github.com/capstone-coal}{Capstone-Coal} organization on github and version control as well as future releases for this project will be managed within that organization.\newline


\noindent \textbf{1.3 Definitions, Acronyms, Abbreviations}\newline


\noindent \href{https://capstone-coal.github.io/team}{COAL-FO}: Coal and Open-pit surface mining impacts on American Lands Follow-On\newline

\noindent \href{https://aviris.jpl.nasa.gov/}{AVIRIS}: Airborne Visible / Infrared Imaging Spectrometer\newline


\noindent \textbf{1.4 Overview}\newline


\noindent This project will, to put it in rough terms, take existing algorithms, improve them, and test the ability to work on other systems. The \href{https://capstone-coal.github.io/team}{COAL-FO} is a continuation of a previous project completed in the 2016-2017 capstone class. The \href{https://capstone-coal.github.io/}{COAL} project was a stable python toolkit providing examples, tests, packages, stable release and stable API that identified, classified, and quantified the effects of open-pit mining on the surrounding environment. Which also included \href{https://github.com/capstone-coal/coal-sds}{COAL-SDS} (Coal Science Data System). Our project will aim at improving the existing algorithms and general functionality as well as enabling the toolkit to work with more spectral libraries. \newline


\noindent {\fontsize{14pt}{14pt}\selectfont \textbf{2. Overall Description}}\newline


\noindent \textbf{2.1 Product Perspective}\newline


\noindent The product will be taken from the coal-capstone github organization and the existing code can potentially be exported to other environments, and used to perform analysis on existing spectral imagery. We will improve the imagery processing algorithms currently in use. We will allow the product to identify more spectra in other spectral libraries such as \href{https://ecosis.org/}{EcoSIS}, \href{https://speclib.jpl.nasa.gov/}{EcoStress} and \href{https://speclab.cr.usgs.gov/spectral-lib.html}{USGS Spectral Library Version 7}.\newline


\noindent \textbf{2.2 Product Functions}\newline


\noindent The product will be functional upon completion, and used on new data that requires analysis. It will be improved pycoal algorithms, general improvements, bug fixes and  \href{https://capstone-coal.github.io/team}{COAL-FO} will also focus on enabling our classifications to use the \href{https://speclab.cr.usgs.gov/spectral-lib.html}{USGS Spectral Library Version 7}, \href{https://ecosis.org/}{EcoSIS} Spectral Library and the \href{https://speclib.jpl.nasa.gov/}{EcoStress} Spectral Library.  \newline


\noindent \textbf{2.3 User Characteristics}\newline


\noindent The user will be able to access this new and improved code from github, and will be able to run it on their own resources that they have available.\newline
\noindent The user will also be able to use the convert functions we will create in mineral.py of pycoal to generate envi .hdr and .sli files from \href{https://speclab.cr.usgs.gov/spectral-lib.html}{USGS Spectral Library Version 7}, \href{https://ecosis.org/}{EcoSIS} Spectral Library and the \href{https://speclib.jpl.nasa.gov/}{EcoStress} Spectral Library. \newline


\noindent \textbf{2.4 Constraints}\newline

\noindent \href{https://aviris.jpl.nasa.gov/}{AVIRIS} data that is available is focused primarily on flight lines while we want to examine areas where coal and open pit surface mining is currently being done or areas which have undergone coal and open pit surface mining in the past. \newline


\noindent \textbf{2.5 Assumptions and Dependencies}\newline


\noindent This project is dependent on the existing \href{https://capstone-coal.github.io/}{COAL} algorithms, the data provided by \href{https://aviris.jpl.nasa.gov/}{AVIRIS} and Spectral Libraries such \href{https://speclab.cr.usgs.gov/spectral-lib.html}{USGS Spectral Library Version 7}, \href{https://ecosis.org/}{EcoSIS} Spectral Library and the \href{https://speclib.jpl.nasa.gov/}{EcoStress} Spectral Library.\newline


\noindent {\fontsize{14pt}{14pt}\selectfont \textbf{3. Specific Requirements}}\newline


\noindent  The requirements of our project will be to improve pycoal algorithms, fix and find bugs existing in pycoal, make general improvements and enable pycoal to work with more spectral libraries specifically \href{https://speclab.cr.usgs.gov/spectral-lib.html}{USGS Spectral Library Version 7}, \href{https://ecosis.org/}{EcoSIS} Spectral Library and the \href{https://speclib.jpl.nasa.gov/}{EcoStress} Spectral Library. More details on the specifics of each of these requirements is provided below. \newline

\href{https://crustal.usgs.gov/speclab/QueryAll07a.php}{USGS Spectral Library Version 7} was release in 2017 and is the newest released version of the \href{https://www.usgs.gov/}{USGS}(United States Geological Survey) spectral library. \href{https://speclab.cr.usgs.gov/spectral.lib06/ds231/index.html}{USGS Spectral Library Version 6} the prior version was released over a decade ago in 2007. \href{https://github.com/capstone-coal/pycoal}{Pycoal} doesn’t use the entire spectral library instead it consumes a convolved library file of the Spectral Library. For \href{https://speclab.cr.usgs.gov/spectral.lib06/ds231/index.html}{USGS Spectral Library Version 6}, \href{https://www.usgs.gov/}{USGS} staged the convolved library files on their \href{ftp://ftpext.cr.usgs.gov/pub/cr/co/denver/speclab/pub/spectral.library/splib06.library/Convolved.libraries/}{website} and the \href{https://capstone-coal.github.io/}{COAL} project was able to use these to identify spectra. The problem \href{https://capstone-coal.github.io/team}{COAL-FO} faces is that much of the documentation used to do this is over a decade old, the spectral libraries are in completely different formats and it doesn’t appear as though \href{https://www.usgs.gov/}{USGS} has plans to stage the convolved library files for \href{https://crustal.usgs.gov/speclab/QueryAll07a.php}{USGS Spectral Library Version 7}.\newline
\newline
The \href{https://capstone-coal.github.io/team}{COAL-FO} project will also aim at using spectra from the \href{https://ecosis.org/}{EcoSIS} and \href{https://speclib.jpl.nasa.gov/}{EcoStress} Spectral Libraries. Enabling pycoal to work with more spectral libraries will allow us to identify more minerals in the \href{https://aviris.jpl.nasa.gov/}{AVIRIS} images. The \href{https://speclib.jpl.nasa.gov/}{EcoStress} Spectral Library Version 1.0 is the spectral library which is built upon the \href{https://speclib.jpl.nasa.gov/downloads/2009-Baldridge.pdf}{ASTER} Spectral Library Version 2.0. Building the convolved library files for this spectral library may not be so difficult since we already have a conversion for the \href{https://speclib.jpl.nasa.gov/downloads/2009-Baldridge.pdf}{ASTER} spectral library which \href{https://speclib.jpl.nasa.gov/}{EcoStress} is adapted from.
\href{https://ecosis.org/}{EcoSIS} Spectral Library has 64k spectra which will be totally new to us. With both \href{https://speclib.jpl.nasa.gov/}{EcoStress} and \href{https://crustal.usgs.gov/speclab/QueryAll07a.php}{USGS Spectral Library Version 7} we are currently using previous versions of these spectral libraries so many of these spectra will have already been identified in previous products we generated with pycoal. \newline
\newline
The specific goal for each one of these Spectral Libraries is to create a convert class in mineral.py or else where in the pycoal subdirectory which will convert these spectral libraries to their convolved library file formats. These formats of the convolved libraries are specifically .sli and .hdr file formats.\newline
\newline
A stretch goal will be for the \href{https://capstone-coal.github.io/team}{COAL-FO} project to crop the default example image. Our client has said this task may take a long time to complete. Therefor, this requirement will be left as a stretch goal if and when the requirements for the spectral libraries mentioned above have been completed, tested and verified to be working correctly. \newline
\newline
Currently, it takes approximately one to two days to run a mineral classification on the default example. The default example is 17.8 gb. By cropping it down to only contain the main areas of interest in this case the flow lines. We could ideally get it down to a size that would only take a few hours to run. This would be very beneficial to new users of pycoal. And would help us a lot with testing. \href{https://aviris.jpl.nasa.gov/}{AVIRIS} images that are 1-2 gb in size currently do exist, and these images typically only take a few hours to run on pycoal, however most of these smaller sized images do not contain lots of activity in terms of minerals identified that we would be interested using as a main example or even for testing purposes. \newline

\noindent {\fontsize{14pt}{14pt}\selectfont \textbf{4. References}}\newline

\noindent [1] COAL-FO, Coal and Open-pit surface mining impacts on American Lands Follow-On\newline \url{https://capstone-coal.github.io/team} \newline

\noindent [2] AVIRIS, Airborne Visible/Infrared Imaging Spectrometer\newline \url{https://aviris.jpl.nasa.gov/} \newline

\noindent [3] AVIRIS-NG, airborne Visible/Infrared Imaging Spectrometer Next Generation\newline \url{https://aviris-ng.jpl.nasa.gov/}\newline

\noindent [4] COAL, Coal and Open-pit surface mining impacts on American Lands\newline \url{https://capstone-coal.github.io/} \newline

\noindent [5] HPC, High Performance Computing\newline \url{https://en.wikipedia.org/wiki/Supercomputer}\newline

\noindent [6] USGS Spectral Library Version 7\newline \url{https://en.wikipedia.org/wiki/Supercomputer}\newline

\noindent [7] EcoSIS Spectral Library \newline \url{https://ecosis.org/}\newline

\noindent [8] EcoSTRESS Spectral Library \newline \url{https://speclib.jpl.nasa.gov/}\newline

\noindent [9] Aster Spectral Library Version 2.0 \newline
\url{https://speclib.jpl.nasa.gov/downloads/2009-Baldridge.pdf}\newline

\end{document}
