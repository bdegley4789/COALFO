% ===============================================
% CS 461: Requirements         Fall 2017
% hw_revised.tex
% Template for revised homework submission
% ===============================================
%         READ THE FOLLOWING CAREFULLY!!!
% ===============================================
% When you produce a PDF version of this document
% to turn in, change the filename to hwX-name.pdf
% replacing X with the homework assignment number
% and name with your last name.
% ===============================================

% -----------------------------------------------
% The preamble that follows can be ignored. Go on
% down to the section that says "START HERE"
% -----------------------------------------------
\documentclass[a4paper,12pt]{article}
\usepackage[colorlinks = true,
            linkcolor = blue,
            urlcolor  = blue,
            citecolor = blue,
            anchorcolor = blue]{hyperref}
\begin{document}
\sloppy

\noindent CS 461 Requirements\newline


\noindent Team Name: \href{http://eecs.oregonstate.edu/capstone/submission/?page=preview\&pid=320}{COAL-FO}\newline


\noindent Bryce Egley, Kenny Thompson\newline


\noindent {\fontsize{14pt}{14pt}\selectfont \textbf{1. Introduction}}\newline


\noindent \textbf{1.1 Purpose}\newline


\noindent The purpose for this document is to layout the requirements for the \href{http://eecs.oregonstate.edu/capstone/submission/?page=preview\&pid=320}{COAL-FO} (COAL Follow On) capstone project and describe how these requirements will be accomplished. \newline

\noindent The purpose of our \href{http://eecs.oregonstate.edu/capstone/submission/?page=preview\&pid=320}{COAL-FO} project is to improve pycoal algorithms, increase code coverage, enable more spectral libraries to work with pycoal, to collect more data to process with pycoal and to improve general functionality of pycoal. \newline

\noindent \textbf{1.2 Scope}\newline


\noindent The project will be focused around publicly available \href{https://capstone-coal.github.io/}{COAL} algorithms created by the previous capstone group, publicly available spectral analysis collected from the NASA \href{https://aviris.jpl.nasa.gov/}{AVIRIS} project. Work for this project will be stored in the \href{https://github.com/capstone-coal}{Capstone-Coal} organization on github and version control as well as future releases for this project will be managed within that organization.\newline


\noindent \textbf{1.3 Definitions, Acronyms, Abbreviations}\newline


\noindent \href{http://eecs.oregonstate.edu/capstone/submission/?page=preview\&pid=320}{COAL-FO}: Coal and Open-pit surface mining impacts on American Lands Follow-On\newline

\noindent \href{https://aviris.jpl.nasa.gov/}{AVIRIS}: Airborne Visible / Infrared Imaging Spectrometer\newline


\noindent \textbf{1.4 Overview}\newline


\noindent This project will, to put it in rough terms, take existing algorithms, improve them, and test the ability to work on other systems. The \href{http://eecs.oregonstate.edu/capstone/submission/?page=preview\&pid=320}{COAL-FO} is a continuation of a previous project completed in the 2016-2017 capstone class. The \href{https://capstone-coal.github.io/}{COAL} project was a stable python toolkit providing examples, tests, packages, stable release and stable API that identified, classified, and quantified the effects of open-pit mining on the surrounding environment. Which also included \href{https://github.com/capstone-coal/coal-sds}{COAL-SDS} (Coal Science Data System). Our project will aim at improving the existing algorithms and general functionality as well as enabling the toolkit to work with more spectral libraries. Increasing the code coverage and creating a  \href{https://restfulapi.net/}{REST API} to run task execution remotely. \newline


\noindent {\fontsize{14pt}{14pt}\selectfont \textbf{2. Overall Description}}\newline


\noindent \textbf{2.1 Product Perspective}\newline


\noindent The product will be taken from the coal-capstone github organization and the existing code can potentially be exported to other environments, and used to perform analysis on existing spectral imagery. We will improve the imagery processing algorithms currently in use. We will allow the product to identify more spectra in other spectral libraries such as \href{https://ecosis.org/}{EcoSIS}, \href{https://speclib.jpl.nasa.gov/}{EcoStress} and \href{https://speclab.cr.usgs.gov/spectral-lib.html}{USGS Spectral Library Version 7}. We will build a \href{https://restfulapi.net/}{REST API} to run task execution remotely for pycoal. \newline


\noindent \textbf{2.2 Product Functions}\newline


\noindent The product will be functional upon completion, and used on new data that requires analysis. It will be improved pycoal algorithms, general improvements, increase code coverage, bug fixes and  \href{http://eecs.oregonstate.edu/capstone/submission/?page=preview\&pid=320}{COAL-FO} will also focus on enabling our classifications to use the \href{https://speclab.cr.usgs.gov/spectral-lib.html}{USGS Spectral Library Version 7}, \href{https://ecosis.org/}{EcoSIS} Spectral Library and the \href{https://speclib.jpl.nasa.gov/}{EcoStress} Spectral Library. As well as a REST API to run task execution remotely. \newline


\noindent \textbf{2.3 User Characteristics}\newline


\noindent The user will be able to access this new and improved code from github, and will be able to run it on their own resources that they have available.\newline
\noindent The user will also be able to use the convert functions we will create in mineral.py of pycoal to generate envi .hdr and .sli files from \href{https://speclab.cr.usgs.gov/spectral-lib.html}{USGS Spectral Library Version 7}, \href{https://ecosis.org/}{EcoSIS} Spectral Library and the \href{https://speclib.jpl.nasa.gov/}{EcoStress} Spectral Library. We will create a REST API using the Falcon Web Framework and deploy using Google Cloud Platform or a similar service. We will test the functionality of this REST API functionality using JSON and Postman. \newline


\noindent \textbf{2.4 Constraints}\newline

\noindent \href{https://aviris.jpl.nasa.gov/}{AVIRIS} data that is available is focused primarily on flight lines while we want to examine areas where coal and open pit surface mining is currently being done or areas which have undergone coal and open pit surface mining in the past. \newline


\noindent \textbf{2.5 Assumptions and Dependencies}\newline


\noindent This project is dependent on the existing \href{https://capstone-coal.github.io/}{COAL} algorithms, the data provided by \href{https://aviris.jpl.nasa.gov/}{AVIRIS} and Spectral Libraries such \href{https://speclab.cr.usgs.gov/spectral-lib.html}{USGS Spectral Library Version 7}, \href{https://ecosis.org/}{EcoSIS} Spectral Library and the \href{https://speclib.jpl.nasa.gov/}{EcoStress} Spectral Library.\newline


\noindent {\fontsize{14pt}{14pt}\selectfont \textbf{3. Specific Requirements}}\newline


\noindent  The requirements of our project will be to improve pycoal algorithms, increase code coverage, fix and find bugs existing in pycoal, make general improvements and enable pycoal to work with more spectral libraries specifically \href{https://speclab.cr.usgs.gov/spectral-lib.html}{USGS Spectral Library Version 7}, \href{https://ecosis.org/}{EcoSIS} Spectral Library and the \href{https://speclib.jpl.nasa.gov/}{EcoStress} Spectral Library. Create a REST API to run task execution remotely.\newline

\noindent {\fontsize{14pt}{14pt}\selectfont \textbf{4. References}}\newline

\noindent [1] COAL-FO, Coal and Open-pit surface mining impacts on American Lands Follow-On\newline \url{http://eecs.oregonstate.edu/capstone/submission/?page=preview\&pid=320} \newline

\noindent [2] AVIRIS, Airborne Visible/Infrared Imaging Spectrometer\newline \url{https://aviris.jpl.nasa.gov/} \newline

\noindent [3] AVIRIS-NG, airborne Visible/Infrared Imaging Spectrometer Next Generation\newline \url{https://aviris-ng.jpl.nasa.gov/}\newline

\noindent [4] COAL, Coal and Open-pit surface mining impacts on American Lands\newline \url{https://capstone-coal.github.io/} \newline

\noindent [5] USGS Spectral Library Version 7\newline \url{https://en.wikipedia.org/wiki/Supercomputer}\newline

\noindent [6] EcoSIS Spectral Library \newline \url{https://ecosis.org/}\newline

\noindent [7] EcoSTRESS Spectral Library \newline \url{https://speclib.jpl.nasa.gov/}\newline

\noindent [8] REST API \newline \url{https://restfulapi.net/}\newline 

\end{document}