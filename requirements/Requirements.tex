% ===============================================
% CS 461: Requirements         Fall 2017
% hw_revised.tex
% Template for revised homework submission
% ===============================================
%         READ THE FOLLOWING CAREFULLY!!!
% ===============================================
% When you produce a PDF version of this document
% to turn in, change the filename to hwX-name.pdf
% replacing X with the homework assignment number
% and name with your last name.
% ===============================================

% -----------------------------------------------
% The preamble that follows can be ignored. Go on
% down to the section that says "START HERE"
% -----------------------------------------------
\documentclass[a4paper,12pt]{article}
\usepackage[colorlinks = true,
            linkcolor = blue,
            urlcolor  = blue,
            citecolor = blue,
            anchorcolor = blue]{hyperref}
\begin{document}
\sloppy

\noindent CS 461 Requirements\newline


\noindent Team Name: COAL-FO\newline


\noindent Bryce Egley, Kenny Thompson\newline


\noindent {\fontsize{14pt}{14pt}\selectfont \textbf{1. Introduction}}\newline


\noindent \textbf{1.1 Purpose}\newline


\noindent The purpose for this document is to layout the requirements for the COAL-FO capstone project and describe how these requirements will be accomplished. \newline

\noindent The purpose of our \href{http://eecs.oregonstate.edu/capstone/submission/?page=preview\&pid=320}{COAL-FO} project as discussed in our last meeting with our client is to to create a searchable port for the existing COAL project. This will be augmented to accommodate the desire to (i) port the coal-sds software to the \href{https://www.xsede.org/}{XSEDE} platform and undertake test and evaluation of the system performance, (ii) process all \href{https://aviris.jpl.nasa.gov/}{AVIRIS} and AVIRIS-NG imagery generating and archiving all science data products, and (iii) making the products searchable through a portal. \newline

\noindent \textbf{1.2 Scope}\newline


\noindent The project will be focused around publicly available \href{https://capstone-coal.github.io/}{COAL} algorithms created by the previous capstone group, publicly available spectral analysis collected from the NASA AVIRIS project, and a grant on the XSEDE environment to use HPC. Work for this project will be stored in the Capstone-Coal organization on github and version control as well as future releases for this project will managed within that organization.\newline


\noindent \textbf{1.3 Definitions, Acronyms, Abbreviations}\newline


\noindent \href{http://eecs.oregonstate.edu/capstone/submission/?page=preview\&pid=320}{COAL-FO}: Coal and Open-pit surface mining impacts on American Lands Follow-On\newline


\noindent HPC: High-performance computing\newline


\noindent \href{https://www.xsede.org/}{XSEDE}: Extreme Science and Engineering Discovery Environment\newline


\noindent \href{https://aviris.jpl.nasa.gov/}{AVIRIS}: Airborne Visible / Infrared Imaging Spectrometer\newline


\noindent \textbf{1.4 References}\newline

IEEE. IEEE Std 830-1998 IEEE Recommended Practice for Software Requirements Specifications. IEEE Computer Society, 1998.\newline


\noindent \textbf{1.5 Overview}\newline


\noindent This project will, to put it in rough terms, take existing algorithms, export them to the XSEDE environment, a high powered computing cluster and process as much publicly available spectro imagery as possible, and use that to create an archive. A ‘stretch goal’ will be to take that archive and make it publicly viewable and searchable. The COAL-FO (COAL Follow On) is a continuation of a previous project completed in the 2016-2017 capstone class. The COAL project was a suite of algorithms that identified, classified, and quantified the effects of open-pit mining on the surrounding environment. Our project will also aim at improving the existing algorithms that were developed in the COAL project and create a new baseline suite to rank changes to areas of land over time. \newline


\noindent {\fontsize{14pt}{14pt}\selectfont \textbf{2. Overall Description}}\newline


\noindent \textbf{2.1 Product Perspective}\newline


\noindent The product will be taken from the coal-capstone github organization and the existing code will be exported to the XSEDE environment, and used to perform analysis on existing spectro imagery. We will improve the imagery processing algorithms currently in use. Then we will create a new baseline suite to rank changes to areas of land over time to determine if the changes in one particular area our significant or not. The main priority will be to create a searchable port for the project which will use the XSEDE environment. \newline


\noindent \textbf{2.2 Product Functions}\newline


\noindent The product will be non functional upon completion, unless further data is generated that requires analysis. The goal will be to create an archive. A stretch goal would be to create a way to search that archive, in which case the functions for that would be a search function and a visual way to represent data. \newline


\noindent \textbf{2.3 User Characteristics}\newline


\noindent The user will be able to access this archived dataset and access processed data to see damage from coal mining in various part of the country. If the stretch goal is reached, then the user will be able to save time by searching a database to get the same information without having to download it first.\newline


\noindent \textbf{2.4 Constraints}\newline


\noindent The XSEDE grant only allowed to 2 terabytes of storage and a certain number of clock hours. To counteract this constraint, we will need to develop a system for rotating in imagery that needs to be processed and out data that's already been processed. We will also need to make sure our algorithms are as efficient as they can possibly be to not waste resources so we can use maximize the use of the 2 terabytes provided to us.\newline


\noindent In addition, we are limited by the AVIRIS data that is available that, while large, barely covers a fraction of the surface of the United States. This data is focused primarily on flight lines and areas where coal and open pit surface mining is currently being done or areas which have undergone coal and open pit surface mining in the past. \newline


\noindent \textbf{2.5 Assumptions and Dependencies}\newline


\noindent This project requires the existing COAL algorithms, the XSEDE environment for rotating in imagery, the data provided to use by AVIRIS and Apache.\newline


\noindent {\fontsize{14pt}{14pt}\selectfont \textbf{3. Specific Requirements}}\newline


\noindent A completed archive, and tools that are exportable to XSEDE. The requirements of our project as discussed with our client is to to create a searchable port for the existing COAL project. Specifically, this will be augmented to accommodate the desire to (i) port the coal-sds software to the XSEDE platform and undertake test and evaluation of the system performance, (ii) process all AVIRIS and AVIRIS-NG imagery generating and archiving all science data products, and (iii) making the products searchable through a portal.\newline

\noindent {\fontsize{14pt}{14pt}\selectfont \textbf{4. References}}\newline

\noindent [1] COAL-FO, Coal and Open-pit surface mining impacts on American Lands Follow-On\newline \url{http://eecs.oregonstate.edu/capstone/submission/?page=preview\&pid=320} \newline

\noindent [2] XSEDE, Extreme Science and Engineering Discovery Environment\newline \url{https://www.xsede.org/} \newline

\noindent [3] AVIRIS, Airborne Visible/Infrared Imaging Spectrometer\newline \url{https://aviris.jpl.nasa.gov/} \newline

\noindent [4] COAL, Coal and Open-pit surface mining impacts on American Lands\newline \url{https://capstone-coal.github.io/}

\end{document}
