% ===============================================
% CS 461: Requirements         Fall 2017
% hw_revised.tex
% Template for revised homework submission
% ===============================================
%         READ THE FOLLOWING CAREFULLY!!!
% ===============================================
% When you produce a PDF version of this document
% to turn in, change the filename to hwX-name.pdf
% replacing X with the homework assignment number
% and name with your last name.
% ===============================================

% -----------------------------------------------
% The preamble that follows can be ignored. Go on
% down to the section that says "START HERE"
% -----------------------------------------------
\documentclass[a4paper,12pt]{article}
\usepackage[colorlinks = true,
            linkcolor = blue,
            urlcolor  = blue,
            citecolor = blue,
            anchorcolor = blue]{hyperref}
\begin{document}
\sloppy

\noindent CS 461 Requirements\newline


\noindent Team Name: \href{http://eecs.oregonstate.edu/capstone/submission/?page=preview\&pid=320}{COAL-FO}\newline


\noindent Bryce Egley, Kenny Thompson\newline


\noindent {\fontsize{14pt}{14pt}\selectfont \textbf{1. Introduction}}\newline


\noindent \textbf{1.1 Purpose}\newline


\noindent The purpose for this document is to layout the requirements for the \href{http://eecs.oregonstate.edu/capstone/submission/?page=preview\&pid=320}{COAL-FO} capstone project and describe how these requirements will be accomplished. \newline

\noindent The purpose of our \href{http://eecs.oregonstate.edu/capstone/submission/?page=preview\&pid=320}{COAL-FO} project as discussed in our last meeting with our client is to to create a searchable port for the existing \href{https://capstone-coal.github.io/}{COAL} project. This will be augmented to accommodate the desire to (i) process all \href{https://aviris.jpl.nasa.gov/}{AVIRIS} and \href{https://aviris-ng.jpl.nasa.gov/}{AVIRIS-NG} imagery generating and archiving all science data products, and (ii) having as a stretch goal, that data available for others to use. \newline

\noindent \textbf{1.2 Scope}\newline


\noindent The project will be focused around publicly available \href{https://capstone-coal.github.io/}{COAL} algorithms created by the previous capstone group, publicly available spectral analysis collected from the NASA \href{https://aviris.jpl.nasa.gov/}{AVIRIS} project, and a grant on the \href{https://www.xsede.org/}{XSEDE} environment to use \href{https://en.wikipedia.org/wiki/Supercomputer}{HPC}. Work for this project will be stored in the Capstone-Coal organization on github and version control as well as future releases for this project will managed within that organization.\newline


\noindent \textbf{1.3 Definitions, Acronyms, Abbreviations}\newline


\noindent \href{http://eecs.oregonstate.edu/capstone/submission/?page=preview\&pid=320}{COAL-FO}: Coal and Open-pit surface mining impacts on American Lands Follow-On\newline


\noindent \href{https://en.wikipedia.org/wiki/Supercomputer}{HPC}: High-performance computing\newline

\noindent \href{https://aviris.jpl.nasa.gov/}{AVIRIS}: Airborne Visible / Infrared Imaging Spectrometer\newline

\noindent \textbf{1.4 References}\newline

IEEE. IEEE Std 830-1998 IEEE Recommended Practice for Software Requirements Specifications. IEEE Computer Society, 1998.\newline


\noindent \textbf{1.5 Overview}\newline


\noindent This project will, to put it in rough terms, take existing algorithms, improve them, and test the ability to work on other systems. The \href{http://eecs.oregonstate.edu/capstone/submission/?page=preview\&pid=320}{COAL-FO} (COAL Follow On) is a continuation of a previous project completed in the 2016-2017 capstone class. The \href{https://capstone-coal.github.io/}{COAL} project was a suite of algorithms that identified, classified, and quantified the effects of open-pit mining on the surrounding environment. Our project will also aim at improving the existing algorithms that were developed in the \href{https://capstone-coal.github.io/}{COAL} project and create a new baseline suite to rank changes to areas of land over time. \newline


\noindent {\fontsize{14pt}{14pt}\selectfont \textbf{2. Overall Description}}\newline


\noindent \textbf{2.1 Product Perspective}\newline


\noindent The product will be taken from the coal-capstone github organization and the existing code can potentially be exported to other environments, and used to perform analysis on existing spectro imagery. We will improve the imagery processing algorithms currently in use. Then we will create a new baseline suite to rank changes to areas of land over time to determine if the changes in one particular area our significant or not.\newline


\noindent \textbf{2.2 Product Functions}\newline


\noindent The product will be non functional upon completion, unless further data is generated that requires analysis. It will be a improved PYCOAL algorithm. \newline
\noindent \href{http://eecs.oregonstate.edu/capstone/submission/?page=preview\&pid=320}{COAL-FO} will also focus on creating conversion functions to allow our classifications to use the \href{https://speclab.cr.usgs.gov/spectral-lib.html}{USGS Spectral Library Version 7}, \href{https://ecosis.org/}{EcoSIS} Spectral Library and the \href{https://speclib.jpl.nasa.gov/}{EcoStress} Spectral Library.  \newline


\noindent \textbf{2.3 User Characteristics}\newline


\noindent The user will be able to access this new and improved code from the github, and will be able to run it on their own resources that they have available.\newline
\noindent The user will also be able to use the convert functions we will create in mineral.py of pycoal to generate .hdr and .sli files from \href{https://speclab.cr.usgs.gov/spectral-lib.html}{USGS Spectral Library Version 7}, \href{https://ecosis.org/}{EcoSIS} Spectral Library and the \href{https://speclib.jpl.nasa.gov/}{EcoStress} Spectral Library. \newline


\noindent \textbf{2.4 Constraints}\newline

\noindent We are limited by the \href{https://aviris.jpl.nasa.gov/}{AVIRIS} data that is available that, while large, barely covers a fraction of the surface of the United States. This data is focused primarily on flight lines and areas where coal and open pit surface mining is currently being done or areas which have undergone coal and open pit surface mining in the past. \newline


\noindent \textbf{2.5 Assumptions and Dependencies}\newline


\noindent This project requires the existing \href{https://capstone-coal.github.io/}{COAL} algorithms, and the data provided to use by \href{https://aviris.jpl.nasa.gov/}{AVIRIS} and Apache.\newline


\noindent {\fontsize{14pt}{14pt}\selectfont \textbf{3. Specific Requirements}}\newline


\noindent  The requirements of our project as discussed with our client is to to create several improvments to the existing \href{https://capstone-coal.github.io/}{COAL} project. Specifically, this will be augmented to accommodate the desire to (i) verify and correct if necessary the COAL-SDS software and its ability to be exported to other platforms, (ii) verify our ability to process all \href{https://aviris.jpl.nasa.gov/}{AVIRIS} and \href{https://aviris-ng.jpl.nasa.gov/}{AVIRIS-NG} imagery, and (iii) making the final product better in several key ways.\newline

\noindent {\fontsize{14pt}{14pt}\selectfont \textbf{4. References}}\newline

\noindent [1] COAL-FO, Coal and Open-pit surface mining impacts on American Lands Follow-On\newline \url{http://eecs.oregonstate.edu/capstone/submission/?page=preview\&pid=320} \newline

\noindent [2] AVIRIS, Airborne Visible/Infrared Imaging Spectrometer\newline \url{https://aviris.jpl.nasa.gov/} \newline

\noindent [3] AVIRIS-NG, irborne Visible/Infrared Imaging Spectrometer Next Generation\newline \url{https://aviris-ng.jpl.nasa.gov/}\newline

\noindent [4] COAL, Coal and Open-pit surface mining impacts on American Lands\newline \url{https://capstone-coal.github.io/} \newline

\noindent [5] HPC, High Performance Computing\newline \url{https://en.wikipedia.org/wiki/Supercomputer}\newline

\noindent [6] USGS Spectral Library Version 7\newline \url{https://en.wikipedia.org/wiki/Supercomputer}\newline

\noindent [7] EcoSIS Spectral Library \newline \url{https://ecosis.org/}\newline

\noindent [8] EcoSTRESS Spectral Library \newline \url{https://speclib.jpl.nasa.gov/}\newline

\end{document}
