% ===============================================
% CS 461: Problem Statement         Fall 2017
% hw_revised.tex
% Template for revised homework submission
% ===============================================
%         READ THE FOLLOWING CAREFULLY!!!
% ===============================================
% When you produce a PDF version of this document
% to turn in, change the filename to hwX-name.pdf
% replacing X with the homework assignment number
% and name with your last name.
% ===============================================

% -----------------------------------------------
% The preamble that follows can be ignored. Go on
% down to the section that says "START HERE"
% -----------------------------------------------

\documentclass{article}

\begin{document}

% ------------------------------------------ %
%                 START HERE                 %
% ------------------------------------------ %

\title{Problem Statement: Coal and Open-pit surface mining impacts on American
Lands Follow-On (COAL-FO)}
\author{Bryce Egley and Kenny Thompson \\CS 461: Senior Capstone}

\maketitle

% -----------------------------------------------------
% The following two environments (theorem, proof) are
% where you will enter the statement and proof of your
% first problem for this assignment.
%
% In the theorem environment, you can replace the word
% "theorem" in the \begin and \end commands with
% "exercise", "problem", "lemma", etc., depending on
% what you are submitting. Replace the "x.yz" with the
% appropriate number for your problem.
%
% If your problem does not involve a formal proof, you
% can change the word "proof" in the \begin and \end
% commands with "solution".
% -----------------------------------------------------

% ---------------------------------------------------------
% Abstract 100-150 Words
%----------------------------------------------------------
\section*{Abstract}
In 2016-2017, three computer science seniors from OSU developed software to
detect the damage caused by coal mining. Their work was impressive, they
developed a suite of algorithms designed to detect and track damage using
satellite imagery. They even managed to win an award and get mentioned in the
OSU alumni newspaper. However there were several promising opportunities to use
their work that was uncompleted at the time of graduation. Our project, and our
goal, is to take their work, improve upon it, and share it for the world to
see. We will be doing it by taking their work and their algorithms, and
converting them into tools that will run on the XSEDE cluster. We will run it
on the XSEDE cluster, feeding it Terabytes of publicly available spectrometer
readings collected from the NASA AVIRIS project. We will then take that data,
and create a search engine to view the processed data in a readable form.
People have created these tools before, and done these analysis’s before, but
it has not been created and made publicly viewable online. This will make
future projects easier for researchers in this field, and will create new and
exciting opportunities for scientific analysis.

\newpage
% ---------------------------------------------------------
% Definition and Description
%----------------------------------------------------------
\section*{Definition and Description}
This project will create new as well as improve upon existing methods to
classify, characterize
and quantify impacts of mining and other destructive surface mining activities
across the United States. The COAL project will now go on XSEDE, a single
virtual system where individuals will be able to share computing resources. By
putting the COAL project on XSEDE others will be able to use the program and
data that comes from the COAL project. This will also enable us to further
improve the algorithm's, execution runtime performance and geospatial output
results. The original COAL project focused on gathering data from remote
sensing devices, providing a suite of algorithms for classifying land cover,
identifying mines and other graphical features, and correlating them with
environmental data sets as described on the coal capstone github page.
\newline
This is the fourth capstone project and hundreds of people have forked the code
from the previous project off github. Since, so many people are using the code
produced in this project it is very important that we make images viewable on
XSEDE so people who want to use the data from this project in the future will
be able to easily gather the data they want and be able to apply it to their
own projects.
\newline
Overall this project is very important since we need to track and monitor the
environmental changes and effects coal has on surrounding lands since Coal is
one of America's main sources of energy. We want to make sure we are extracting
coal and using it safely for the sake of the environment. Satellite imaging
using remote sensing devices allows us to gather much more data than a human
eye is able to. Satellite imaging can provide a picture of an entire area of
land and see things invisible to the human eye such as methane emissions.


% ---------------------------------------------------------
% Proposed Solution
%----------------------------------------------------------
\section*{Proposed Solution}
So, our problem is, that to get this data which is essential to many research
projects, researchers have to create tools and process the data, sometimes
processing data that’s already been done. The first half has already largely
been solved, thanks to the 2016-2017 capstone team. The real issue is the
second part. In order to solve the problem addressed. we wish to put the
COAL-FO (COAL follow on) project on XSEDE, a high powered computing cluster,
and analyze as much publicly available information as possible. XSEDE only
allows for 2 terabytes so we will come up with a method to feed it data that
will be looped through. The main priority for our capstone project is making a
searchable port for the COAL project. We also plan on having meetings every two
weeks where we will turn in and discuss past week deliverables and plan future
deliverables. This is where we will create a schedule of how to complete the
project. As already discussed we will also plan on improving the existing COAL
architecture and systems. We also hope to improve the imagery processing
algorithms currently in use. Then we will hope to create a new baseline suite
to rank changes to areas of land over time to determine if the changes in one
particular area our significant or not.

% ---------------------------------------------------------
% Performance Metrics
%----------------------------------------------------------
\section*{Performance Metrics}
The performance metrics for this capstone project will be to complete as many
of the objectives as possible discussed in the abstract and the project
description. Our main priority according to our client is to create a
searchable port for the project. Once we complete this we will move on to the
other objectives. We will also have bi-weekly performance metrics based on the
meetings we have with our client and the deliverables that come from those
meetings. If we can complete those objectives and the high priority objectives
then this project will be a success.
\newline
This means we don't necessarily need to complete all the objectives for this
project to be successful. As long as the high priority objectives like the
searchable port and maybe putting the project on XSEDE are completed. The other
objectives should only be given focus once we have completed work on the high
priority objectives. These lower priority objectives would include improving
existing algorithms and improving the accuracy of finding correlations between
land changes in different images.
\newline
Overall this project looks challenging and it seems like a lot of work but if
we focus on one objective at a time and use the resources available to us I
believe we can accomplish several of the high priority objectives. We also hope
that the work we do on this project will make it more accessible and usable for
other individuals working on their own projects and research. We look forward
to working with the NASA Jet Propulsion Lab as well as the other members of the
COAL project from previous years.


% -----------------------------------------------
% Ignore everything that appears below this.
% -----------------------------------------------

\end{document}
