\documentclass[10pt,draftclsnofoot,onecolumn,journal,compsoc]{IEEEtran}
% for IEEEtran usage, see http://www.texdoc.net/texmf-dist/doc/latex/IEEEtran/IEEEtran_HOWTO.pdf

\usepackage[margin=0.75in]{geometry}
\usepackage{graphicx}
\usepackage{enumerate}
\usepackage{amssymb}

\renewcommand{\linespread}{1.0}
\usepackage[colorlinks = true,
            linkcolor = blue,
            urlcolor  = blue,
            citecolor = blue,
            anchorcolor = blue]{hyperref}

\title{CS 461 Technology Review and Implementation Plan}
\author{Bryce Egley and Kenny Thompson \\ Group Number: 28 \\ Project Name: \href{http://eecs.oregonstate.edu/capstone/submission/?page=preview\&pid=320}{COAL-FO} \\ Our Roles: Developers}
\date{}
\begin{document}
\maketitle
  \begin{abstract}
	\href{https://capstone-coal.github.io/}{COAL} and Open-pit surface mining impacts on American Lands Follow-On (COAL-FO) is the successor project to the 2016-
2017 \href{https://capstone-coal.github.io/}{COAL} project. \href{https://capstone-coal.github.io/}{COAL} initially aimed to deliver a suite of algorithms to identify, classify, characterize, and quantify (by reporting a
number of key metrics) the direct and indirect impacts of mining operations and related destructive surface mining activities across the
continental U.S (and further afield). \href{https://capstone-coal.github.io/}{COAL} successfully delivered a Python library for processing hyperspectral imagery from remote
sensing devices such as the Airborne Visible/InfraRed Imaging Spectrometer \href{https://aviris.jpl.nasa.gov/}{AVIRIS} and a Science Data System for running \href{https://capstone-coal.github.io/}{COAL}
pipelines. \href{http://eecs.oregonstate.edu/capstone/cs/capstone.cgi?project=320}{COAL-FO} will utilize recent funding obtained from a recently awarded NSF-funded \href{https://www.xsede.org/}{XSEDE} high performance computing
(HPC) grant to further improve, validate and document \href{https://capstone-coal.github.io/}{COAL} algorithms, execution runtime performance and geospatial output
results.\cite{1}
  \end{abstract}
\newpage

\tableofcontents

\newpage

\section{Introduction}

\subsection{What we are trying to accomplish}

\noindent The purpose of our \href{http://eecs.oregonstate.edu/capstone/cs/capstone.cgi?project=320}{COAL-FO} project as discussed in our last meeting with our client is to to create a searchable port for the existing \href{https://capstone-coal.github.io/}{COAL} project. This will be augmented to accommodate the desire to (i) port the coal-sds software to the \href{https://www.xsede.org/}{XSEDE} platform and undertake test and evaluation of the system performance, (ii) process all \href{https://aviris.jpl.nasa.gov/}{AVIRIS} and \href{https://aviris-ng.jpl.nasa.gov/}{AVIRIS-NG} imagery generating and archiving all science data products, and (iii) making the products searchable through a portal. \newline

\noindent The project will be focused around publicly available \href{https://capstone-coal.github.io/}{COAL} algorithms created by the previous capstone group, publicly available spectral analysis collected from the NASA \href{https://aviris.jpl.nasa.gov/}{AVIRIS} project, and a grant on the \href{https://www.xsede.org/}{XSEDE} environment to use \href{https://en.wikipedia.org/wiki/Supercomputer}{HPC}. Work for this project will be stored in the Capstone-Coal organization on github and version control as well as future releases for this project will managed within that organization.

\subsection{Project Overview}

\noindent This project will, to put it in rough terms, take existing algorithms, export them to the \href{https://www.xsede.org/}{XSEDE} environment, a high-powered computing cluster and process as much publicly available spectro imagery as possible, and use that to create an archive. A ‘stretch goal’ will be to take that archive and make it publicly viewable and searchable. The \href{http://eecs.oregonstate.edu/capstone/cs/capstone.cgi?project=320}{COAL-FO} (COAL Follow On) is a continuation of a previous project completed in the 2016-2017 capstone class. The \href{https://capstone-coal.github.io/}{COAL} project was a suite of algorithms that identified, classified, and quantified the effects of open-pit mining on the surrounding environment. Our project will also aim at improving the existing algorithms that were developed in the \href{https://capstone-coal.github.io/}{COAL} project and create a new baseline suite to rank changes to areas of land over time. \newline

\noindent The product will be taken from the \href{https://capstone-coal.github.io/}{coal-capstone} GitHub organization and the existing code will be exported to the \href{https://www.xsede.org/}{XSEDE} environment, and used to perform analysis on existing spectro imagery. We will improve the imagery processing algorithms currently in use. Then we will create a new baseline suite to rank changes to areas of land over time to determine if the changes in one area our significant or not. The main priority will be to create a searchable port for the project which will use the \href{https://www.xsede.org/}{XSEDE} environment.

\section{Technologies}

\subsection{HPCC High Performance Computing Cluster}

\subsubsection{Options}

\noindent I Enterprise Service Platform \newline
\noindent II OSU Unix HPC Cluster \newline
\noindent III XSEDE

\subsubsection{Goals}

\noindent Our goals for this portion of the project is too make the data easier to view and less costly to users. We are limited by the AVIRIS data that is available that, while large, barely covers a fraction of the surface of the United States. This data is focused primarily on flight lines and areas where coal and open pit surface mining is currently being done or areas which have undergone coal and open pit surface mining in the past. Our goal is to set up an HPCC system which can process and store the AVIRIS images.

\subsubsection{Criteria}

\noindent We want an HPCC to store AVIRIS images so users don’t need to use their own personal computing resources to view and collect data from the AVIRIS images. The system will need to have several terabytes of storage and have a low cost.

\subsubsection{Comparison}

\noindent Out of the three options for HPCC discussed so far only XSESE has been already paid for and we were allocated 2 terabytes of storage. We may be able to use OSU Unix HPC Center as part of our tuition but with XSEDE we would have a more permanent place to store data since we will graduate and our storage through an OSU system would expire upon graduation. Enterprise Service Platform is a good HPCC however it will cost us money to use and we have already been given a grant to use XSEDE.

\subsubsection{Discussion}

\noindent XSEDE is the Extreme Science and Engineering Discovery Environment. XSEDE will allow us to use 2 Tera-bytes for processing images. We will loop images through XSEDE and have them stored on another database for other users to then access them on the XSEDE platform. This will save other users a lot of time since they won’t have to deal with generating the images each time and will just have the images ready to go on XSEDE. We hope this will make the COAL capstone project from the previous year and continuation with our project this year can reach a wider audience and be used in more research for people studying the effects of coal on American lands. Putting our program on the XSEDE platform is the main objective of our project as outlined by our client. Our group has been given a grant to do this and the 2 terabytes on XSEDE for now, although we could be given more money and space in the future.

\subsubsection{Selection}

\noindent XSESDE

\subsection{Database Language}

\subsubsection{Options}

\noindent I NoSQL \newline

\noindent II SuprTool \newline

\noindent III MySQL

\subsubsection{Goals}

\noindent We wish to use a database language so that we can loop images through XSEDE and store other data for this project in a database and have it managed all in one place. This will make it easier to design and organize our project.

\subsubsection{Criteria}

\noindent This will be the language we use to manage our database which will hold the AVIRIS-NG imagery before we loop it through XSEDE. We want a database language that we are all familiar with and will work for the data we are working with. MySQL was the database language introduced to us in Introduction to Databases(cs340) it is relatively easy to implement this database language with existing code given to us by the COAL project from 2016-2017. NoSQL is used in non-relational databases and it uses key values to keep track of data. SuprTool is a database language used in Oracle databases for images.

\subsubsection{Comparison}

\noindent NoSQL is a good database language however our project will involve a relational database so it won’t work for this case. SuprTool is also good for images which we are using, however we aren’t using an Oracle database so it won’t work either. MySQL was a programming language developed in in the early 90’s. It is a widely-used programming language for databases and is used in sites such as Google, Facebook, YouTube, Twitter. The language was developed in C and C++ and can run on many different operating systems such as Windows, Free BSD, mac OS, Linux etc.

\subsubsection{Discussion}

\noindent MySQL is also supported by Oregon State and all engineering students and we are given our own database for storing data on. The only caveat to this is that all data stored on the Oregon State engineering student databases is that they are deleted at the end of each term, so we will need to make sure we can save this data somewhere to be later accessed and the reloaded to the engineering student database. Or to another database somewhere else to access at another time. The goal with XSEDE is to have all the images saved on one database then researchers or engineers can loop through and look at our work and decide what they want to use or view without having to go through the tedious process of downloading each image individually and running our code each time they want an image. If we can simply run the program ourselves and then get all the images generated onto a database which can then be downloaded later by researchers and engineers.

\subsubsection{Selection}

\noindent MySQL

\subsection{License}

\subsubsection{Options}

\noindent I MIT License \newline
\noindent II GNU GPLv2 \newline
\noindent III Apache License v2.0

\subsubsection{Goals}

\noindent Our goal is to have a license that will allow us to maintain ownership and credit for our code but also keep it open source and available for other research projects. We want a license that is free and easy to use.

\subsubsection{Criteria}

\noindent We need a license for our code that won't require too much change to the existing code and won't be heavy in documentation. A lot of our current code uses the GNU GPLv2 and Apache License v2.0. So, we would like to use these licenses or licenses very similar to them.

\subsubsection{Comparison}

\noindent The MIT License is the most popular software license as of 2015\cite{19}. This gives the MIT License an advantage over other open source licenses we may consider using, since it would give us good experience using an open source license which we will probably use again in the future. It should also meet most of our criteria since a license that is very popular will likely be easy to use and do a good job at allowing us to maintain ownership of our code. The MIT License is also compatible with GNU which is good since some of the existing code for our project is already under the GNU GPLv2 License.
\newline
\noindent GNU 2.0 General Purpose 2.0 License is very easy to share and use. It also the second most common open source license as of 2015\cite{19}. This like the MIT License would give us good experience since it is very likely that we will use it in the future.
\newline
\noindent Apache License v2.0 shares the same benefits as the two other open source licenses discussed previously. It is the third most common open source licenses as of 2015\cite{19}.

\subsubsection{Discussion}

\noindent Since our existing code is already covered under the GNU GPLv2 for pycoal and Apache License v2.0 for coal-sds we will continue to use these open source licenses on the code we develop for our capstone project.

\subsubsection{Selection}

\noindent  Apache License v2.0 and GNU GPLv2

\subsection{Imagery}

\subsubsection{Options}

\noindent I Google Maps \newline
\noindent II AVIRIS \newline
\noindent III US geological survey data

\subsubsection{Goals}
\noindent The goal for this section is to process as much publicly available data as possible.

\subsubsection{Criteria}
\noindent The data we are going to
read will need to be able to detect damage from coal mining (the presence of heavy metals in the rivers and
surrounding areas, etc) and must be conducted in areas where coal mining has occurred, and also in areas where
coal mining hasn’t occurred, so we can get a baseline.

\subsubsection{Comparison}
AVIRIS was what the client requested, but there are pros and cons for using it. For one thing, it only covers
certain sections of land, a fraction of the US landmass. It also isn't ideal for the stated goal, which is analyzing
coal mining damage. Since it only covers certain areas, we arent getting a complete picture of coal mining. The
pros are, the technology already developed works well with AVIRIS, since it was specifically designed to analyze
spectro-imagery. This meets the first criteria well, but is not great on the second criteria, because as earlier stated,
it is very limited in the area it covers. While not ideal, this will still work because on the area it did cover we
can extrapolate to make assumptions about other areas where that type of mining was conducted.\cite{3}
A possible alternative would be google maps. Google maps includes all imagery nationwide, although some
of it is less current and less high resolution than others. Google earth also gives us a host of other data that other
sources do not have, and its possible we could use all this data to extrapolate coal damage in all sorts of ways.
For example, Google maps shows the activity of foot traffic in certain areas, and what time of day that foot traffic
is highest. This certainly meets our second criteria in spades. Not to mention Google is already in a searchable
easy to manipulate database. Unfortunately, google maps does not include spectro analysis, and would not work
with the technology we have developed. The quality of photography just isnt enough to determine damage from
things like spillage and contamination. So while you could certainly extrapolate using other methods, it will
likely be less high quality of analysis compared to actual imagery on the ground.
Another possibility is US geological survey data. This is data that is published online from the USGS and
thanks to government funding is far reaching. This source includes a lot of data not included in either AVIRIS or
google maps, and has live feeds coming from all across the US. Its free to use and trustworthy, and has access to
resources and datasets that no one else can match, it even has datasets of spectro imagery so you can figure out
what you are looking at. It does not meet either criteria well, since its not all in the same easy to process format
as AVIRIS and Google maps it would be harder to make simple code that can analyze all of it at once. \cite{2}

\subsubsection{Discussion}
We will be using AVIRIS. Since the data is in the format we already want, and its free and publicly accessible,
we will be able to use it to meet both criteria. We will also be making use of some of the datasets in the USGS
database so we can determine what we are looking at.

\subsubsection{Selection}
\noindent AVIRIS

\subsection{Export/Import Tool}

\subsubsection{Options}
\noindent I Globus Command Line Interface \newline
\noindent II Regular Globus \newline
\noindent III Globus url copy and uberftp

\subsubsection{Goals}
\noindent For this project, we will be using a high powered computing cluster. We will need a tool to import data from
our image analysis to the server, and a tool to export the processed data onto a computer.

\subsubsection{Criteria}
\noindent Only a certain amount
of data can be stored, but we cant stop every time it fills up to manually run it, so we will need this tool to run
automatically so things can run smoothly.

\subsubsection{Comparison}
The Globus command line interface is a tool to export and import data out of XSEDE. Globus command line
interface is a little harder to use than some other ones, it requires a ssh key and some additional setup procedures,
as well as just advanced knowledge on databases to begin with. However, its easier to automate, and since we
will be dealing with massive amounts of data, automation is a good thing. \cite{4}
A possible alternative would be regular Globus. Globus is easier to use, has a web interface, can be set up to
use just a simple login, and there is even a desktop download to run it from your computer. The downside is its
less easy to automate so a lot of the work would have to be done manually, which when we are dealing with
this amount of data tends to be a problem unless you want to manually move everything over constantly.
Another possible alternative would be globus url copy and uberftp. This has the same drawbacks as globus
command line interface, in that its harder to use, requires advanced knowledge, and requires additional authentication
procedures, however this is more designed for high performance jobs, and is more efficient than the other
two.

\subsubsection{Discussion}
\noindent For the amount of data we are dealing with, and the amount of wiggle room we have with the data they've
given us, this is not required and is not an issue.

\subsubsection{Selection}
None

\subsection{Programming Language}

\subsubsection{Options}
\noindent I Python \newline
\noindent II C++ \newline
\noindent III Java

\subsubsection{Goals}
\noindent The goal for this section is to have code that scans spectro-imagery and determines damage.

\subsubsection{Criteria}
\noindent We used python
for the reason that the program was already set up in python and thats what the client wanted, but python is
still the correct tool for this scenario because it is a general purpose language that can be used to both take in
the imagery, run analysis on it, and pump out readable results. This might require 2 or even 3 programming
languages if you tried to do it without python.

\subsubsection{Comparison}
\noindent Another possibility is C++. C++ would be a good option because its easier to run on XSEDE and works well
with GLOBUS, its also a program more people are familiar with. However its limited flexibility would require
another programming language or some 3rd party tool to convert spectro-imagery into a format its able to read.\cite{5}
Java would also be a good possibility, because it works well with databases, and might make it easier to
export/import data from other sources. However the tooling would require significant work to take the functions
we have on Python and make them work on Java, and even then it might not work as well as the existing
applications.

\subsubsection{Discussion}
\noindent We will be using python for the reasons stated above.

\subsubsection{Selection}
\noindent Python

\section{Conclusion}

\noindent In this document, we covered six different technologies we will be using to implement and design our COAL-FO capstone project. We discussed pros and cons of different pieces for each technology we would be using. When we start implementation of the project we may change design choices as certain pros and cons may not have been thought of and adequately represented in our planning. We hope we can in general stick to the technology choices we made in this document.

\begin{thebibliography}{9}
\bibitem{1} ”CS461 - CS Senior Capstone”, Eecs.oregonstate.edu, 2017. [Online]. Available: \url{http://eecs.oregonstate.edu/capstone/cs/capstone.cgi?project=320} [Accessed: 22- Nov- 2017]

\bibitem{2} ”USGS.gov — Science for a changing world”, Usgs.gov, 2017. [Online]. Available: \url{https://www.usgs.gov/} [Accessed: 22- Nov- 2017]

\bibitem{3} ]”AVIRIS - Airborne Visible / Infrared Imaging Spectrometer”, Aviris.jpl.nasa.gov, 2017. [Online]. Available:
\url{https://aviris.jpl.nasa.gov/} [Accessed: 22- Nov- 2017].

\bibitem{4} ”XSEDE User Portal — Globus User Guide”, Portal.xsede.org, 2017. [Online]. Available: \url{https://portal.xsede.org/software/globus} [Accessed: 22- Nov- 2017].

\bibitem{5} ”C++”, En.wikipedia.org, 2017. [Online]. Available:
\url{https://en.wikipedia.org/wiki/C} [Accessed: 22- Nov- 2017].

\bibitem{6} "MySQL", En.wikipedia.org, 2017. [Online]. Available: \url{https://en.wikipedia.org/wiki/MySQL} [Accessed: 22- Nov- 2017].

\bibitem{7} "XSEDE, Extreme Science and Engineering Discovery Environment", www.xsede.org, 2017. [Online]. Available: \url{https://www.xsede.org/} [Accessed: 22- Nov- 2017].

\bibitem{8} "AVIRIS, Airborne Visible/Infrared Imaging Spectrometer", aviris.jpl.nasa.gov, 2017. [Online]. Available: \url{https://aviris.jpl.nasa.gov/} [Accessed: 22- Nov- 2017].

\bibitem{9} "AVIRIS-NG, airborne Visible/Infrared Imaging Spectrometer Next Generation", aviris-ng.jpl.nasa.gov, 2017. [Online]. Available: \url{https://aviris-ng.jpl.nasa.gov/}[Accessed: 22- Nov- 2017].

\bibitem{10} "COAL, Coal and Open-pit surface mining impacts on American Lands", capstone-coal.github.io, 2017. [Online]. Available: \url{https://capstone-coal.github.io/} [Accessed: 22- Nov- 2017].

\bibitem{11} "HPC, High Performance Computing", en.wikipedia.org, 2017. [Online]. Available: \url{https://en.wikipedia.org/wiki/Supercomputer} [Accessed: 22- Nov- 2017].

\bibitem{12} "Query Language", en.wikipedia.org, 2017. [Online]. Available: \url{https://en.wikipedia.org/wiki/Query_language} [Accessed: 22- Nov- 2017].

\bibitem{13} "Enterprise Module Service", hpccsystems.com, 2017. [Online]. Available: \url{https://hpccsystems.com/enterprise-services/modules/esp} [Accessed: 22- Nov- 2017].

\bibitem{14} "HPCC High Performance Computing Cluster", hpccsystems.com, 2017. [Online]. Available: \url{https://hpccsystems.com/enterprise-services/modules/esp} [Accessed: 22- Nov- 2017].

\bibitem{15} "OSU Unix HPC Cluster", cosine.oregonstate.edu, 2017. [Online]. Available \url{http://cosine.oregonstate.edu/unix-hpc-cluster} [Accessed: 22- Nov- 2017].

\bibitem{16} "MIT License", en.wikipedia.org, 2017. [Online]. Available: \url{https://en.wikipedia.org/wiki/MIT_License} [Accessed: 29- Nov- 2017].

\bibitem{17} "Apache License Version 2.0", www.apache.org, 2017. [Online]. Available \url{https://www.apache.org/licenses/LICENSE-2.0} [Accessed: 29- Nov- 2017].

\bibitem{18} "GNU General Public License, version 2", www.gnu.org, 2017. [Online]. Available \url{https://www.gnu.org/licenses/old-licenses/gpl-2.0.en.html} [Accessed: 29- Nov- 2017].

\bibitem{19} "Top Open Source Licenses",www.blackducksoftware.com, 2017.[Online]. Available \url{https://www.blackducksoftware.com/top-open-source-licenses} [Accessed: 29- Nov- 2017].
\end{thebibliography}
\end{document}
