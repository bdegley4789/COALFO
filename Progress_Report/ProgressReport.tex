\documentclass[onecolumn, draftclsnofoot,10pt, compsoc]{IEEEtran}
\usepackage{graphicx}
\usepackage{url}
\usepackage{setspace}

\usepackage{geometry}
\geometry{textheight=9.5in, textwidth=7in}
\bibliographystyle{IEEEtran}

% 1. Fill in these details
\def \CapstoneTeamName{		COALFO}
\def \CapstoneTeamNumber{		28}
\def \GroupMemberOne{			Kenny Thompson}
\def \GroupMemberTwo{			Bryce Egley}
\def \CapstoneProjectName{		Turning spectro-imagery into usable data}
\def \CapstoneSponsorCompany{	NASA JPL}
\def \CapstoneSponsorPerson{		Lewis John Mcgibbney}

% 2. Uncomment the appropriate line below so that the document type works
\def \DocType{	%Problem Statement
				%Requirements Document
				%Technology Review
				%Design Document
				Progress Report
				}

\newcommand{\NameSigPair}[1]{\par
\makebox[2.75in][r]{#1} \hfil 	\makebox[3.25in]{\makebox[2.25in]{\hrulefill} \hfill		\makebox[.75in]{\hrulefill}}
\par\vspace{-12pt} \textit{\tiny\noindent
\makebox[2.75in]{} \hfil		\makebox[3.25in]{\makebox[2.25in][r]{Signature} \hfill	\makebox[.75in][r]{Date}}}}
% 3. If the document is not to be signed, uncomment the RENEWcommand below
\renewcommand{\NameSigPair}[1]{#1}

%%%%%%%%%%%%%%%%%%%%%%%%%%%%%%%%%%%%%%%
\begin{document}
\begin{titlepage}
    \pagenumbering{gobble}
    \begin{singlespace}
    	%\includegraphics[height=4cm]{coe_v_spot1}
        \hfill
        % 4. If you have a logo, use this includegraphics command to put it on the coversheet.
        %\includegraphics[height=4cm]{CompanyLogo}
        \par\vspace{.2in}
        \centering
        \scshape{
            \huge CS Capstone \DocType \par
            {\large\today}\par
            \vspace{.5in}
            \textbf{\Huge\CapstoneProjectName}\par
            \vfill
            {\large Prepared for}\par
            \Huge \CapstoneSponsorCompany\par
            \vspace{5pt}
            {\Large\NameSigPair{\CapstoneSponsorPerson}\par}
            {\large Prepared by }\par
            Group\CapstoneTeamNumber\par
            % 5. comment out the line below this one if you do not wish to name your team
            \CapstoneTeamName\par
            \vspace{5pt}
            {\Large
                \NameSigPair{\GroupMemberOne}\par
                \NameSigPair{\GroupMemberTwo}\par
            }
            \vspace{20pt}
        }
        \begin{abstract}
        % 6. Fill in your abstract
        	Coal and Open-pit surface mining impacts on American Lands Follow-On (COAL-FO) is the successor 				project to the 2016-2017 COAL project. COAL initially aimed to deliver a suite of algorithms to identify, classify, characterize, and quantify (by reporting a number of key metrics) the direct and indirect impacts of mining operations and related destructive surface mining activities across the continental U.S (and further afield). COAL successfully delivered a Python library for processing hyperspectral imagery from remote sensing devices such as the Airborne Visible/InfraRed Imaging Spectrometer (AVIRIS) and a Science Data System for running COAL pipelines. COAL-FO will utilize recent funding obtained from a recently awarded NSF-funded XSEDE high performance computing (HPC) grant to further improve, validate and document COAL algorithms, execution runtime performance and geospatial output results.[1]
        \end{abstract}
    \end{singlespace}
\end{titlepage}
\newpage
\pagenumbering{arabic}
\tableofcontents
% 7. uncomment this (if applicable). Consider adding a page break.
%\listoffigures
%\listoftables
\clearpage

% 8. now you write!
\section{Project Purpose and Goals}

COAL and Open-pit surface mining impacts on American Lands Follow-On (COAL-FO) is the successor project to
the 2016- 2017 COAL project. COAL initially aimed to deliver a suite of algorithms to identify, classify, characterize,
and quantify (by reporting a number of key metrics) the direct and indirect impacts of mining operations and related
destructive surface mining activities across the continental U.S (and further afield). COAL successfully delivered a
Python library for processing hyperspectral imagery from remote sensing devices such as the Airborne Visible/InfraRed
Imaging Spectrometer AVIRIS and a Science Data System for running COAL pipelines. COAL-FO will utilize recent
funding obtained from a recently awarded NSF-funded XSEDE high performance computing (HPC) grant to further
improve, validate and document COAL algorithms, execution runtime performance and geospatial output results.[1]

\section{Where we are on the project}

We have mostly been focusing on completing the administrative tasks involved, ie the design document, the tech review, as well as limited testing and broad discussions on the desired scale and shape of our project. As we transition out of fall term and into winter term, we are now focusing on more extensive testing and actual implementation of the discussed ideas. We will also be focusing on making the code more efficient and making our end product more useful to the public.

\section{Problems impeding progress}
These are unassigned issues left over from the existing COAL capstone code. \newline
1. Improve QGIS Install/Usage Instructions - Many of the instructions for installing the existing are hard to follow and prone to errors \newline
2. Convert all code to Cython - Cython is a superset of the Python that has major speed improvements over Python. \newline
3. Our client is currently implementing a REST API for our project
Label Docker Image - improve interpretation of the Docker image by labeling it. \newline
4. Get More Data - Collect data of another coal mine, or a region contained within the USGS Spectral Library Version 7. The data for the existing project is focused on a San Juan Mine case study.
5. Label Docker Image - improve interpretation of the Docker image by labeling it.

\subsection{Solutions?}

The solution to these issues will be to try to tackle them over Winter break. Bryce plans to work on 1 Improving QGIS Install and 2 Converting code to Cython, Kenny can try to solve 4 Getting more Data and 5 Labeling the Docker Image. In the meanwhile our client Lewis will work on getting the REST API set up.

\section{Retrospective of the past 10 weeks}

\begin{center}
\begin{tabular}{ p{0.3\linewidth} p{0.3\linewidth} p{0.3\linewidth} }
 Positives & Deltas & Actions \\
 Completed Design Document  & Need to convert code to usable form & Will convert code \\
 Completed Tech Review & Need to set up export/import of data & Will set up \\
 Completed Requirements Doc & Need to figure out database structure & Will review with client then set up \\
 Completed Problem Statement & Need to make code more efficient & Will begin testing and analyzing code
\end{tabular}
\end{center}

\section{Week by week summary}

\subsection{Week 1}

\paragraph{Plan}
This week we plan to choose the project preferences.
\paragraph{Problems}
After reviewing the project preferences list there were several projects I didn't think I would have the required knowledge for.
\paragraph{Progress}
I have made my selection on my project preferences.
\paragraph{Summary}
For our first week reviewed the projects that we wanted, developed our preferences list, and set up our onenotes to include our biographies pages.

\subsection{Week 2}
\paragraph{Plan}
This week I planned to read into my capstone project once I had been assigned and to contact my client.
\paragraph{Problems}
 I still didn't know what my project was.
\paragraph{Progress}
Now we have been assigned to our Project. Our project is Prison Match.
\paragraph{Summary}
For the second week, we got our project, reached out to our client, and coordinated a meeting time. We also got to know our team and developed systems of communication we would use throughout our project.

\subsection{Week 3}
\paragraph{Plan}
 This week our plan was to have a meeting with our client to get more details on our project.
\paragraph{Problems}
Our project was reassigned because our old client for the Prison Match project didn't want to do a capstone project this year. Also one of our group members dropped the course so now we only have two people.
\paragraph{Progress}
On Thursday we were able to have a google hangouts meeting with our client. We went over deliverables and the main objectives for our project.
\paragraph{Summary}
This week we were reassigned from the prison match project to  Coal and Open-Pit surface mining impacts on American Lands Follow-On our client is Lewis McGibbney who works for NASA JPL. We worked on our problem statement and complete a rough draft individually. We would also conduct our first meeting with our client and established procedures that we would follow for client communication for the remainder of the time we worked on our project.

\subsection{Week 4}
\paragraph{Plan}
This week my partner and I want to finish and send our problem statement to our client.
\paragraph{Problems}
In order to finish our problem statement we need to combine the individual problem statements we wrote up into one problem statement which we will send to our client.
\paragraph{Progress}
On Tuesday we put together our problem statemnts and edited it for errors. We then sent our problem statement to our client in an email to sign off on. We cc'd the professors and the TA in the problem statement email. Our Client, Lewis replied the same night and checked off on our problem statement.
\paragraph{Summary}
This week we worked on our problem statement and discussed it with our client. He had some minor corrections to make on it, but after that we turned it in and got the client to verify. We met with our client and again discussed the future of this project, and got his input on some upcoming documents we were working on.

\subsection{Week 5}
\paragraph{Plan}
This week my plan is to create the rough draft of requirements document.
\paragraph{Problems}
The problem for this is week is that I'm still not sure which parts of the capstone my client wants in the requirement. We will discuss this with him on Thursday in our weekly meeting.
\paragraph{Progress}
This week we met with our client and created the rough draft of the requirements document.
\paragraph{Summary}
This week we turned in our Requirements document rough draft. We discussed it at length with our client and again implemented some minor corrections.

\subsection{Week 6}
\paragraph{Plan}
This week we planned to complete the requirements document and send it to our client for approval.
\paragraph{Problems}
We need to come up with the requirements for the requirements document due on Friday that we need to send to our client.
\paragraph{Progress}
As of Wednesday we completed our requirements document and sent it to our client to look over and approve. We already had the rough draft on our github. On Friday our client approved our requirements document.
\paragraph{Summary}
This week we emailed our client for his final input and approval of our requirements document after telling us to add a reference section. After he gave his approval we turned it in.

\subsection{Week 7}
\paragraph{Plan}
This week we planned to have a meeting with our client on Thursday.  We also planned to start the technology review assignment.
\paragraph{Problems}
This week we met with our client and discuss some more specifics on the capstone coal project. Specifically what deliverables he wants for the week 9 meeting.
\paragraph{Progress}
This week I've worked on the technology review and discussed with my partner whether or not we should have a meeting in the Thursday in two weeks since that will be Thanksgiving Day.
\paragraph{Summary}
This week we went to class and worked on the code on our own. We also worked on thje technology review. We had a meeting with our client where we discussed a few pressing issues and scheduled around thanksgiving week.

\subsection{Week 8}
\paragraph{Plan}
This week we plan to write up the Technology Review rough draft, choose issues to work on in the github for our project COAL-FO and run the existing code.
\paragraph{Problems}
The problem is that we need to create the rough drafts for our technology review and start choosing issues to work on.
\paragraph{Progress}
This week I ran the tests for our existing code but I'm still working on running the example problems. I also wrote up the technology review document.
\paragraph{Summary}
This week we turned in our technology review rough draft and conducted peer reviews in class on the documents. We would then go on to work on our final draft, and prepared the information we would need for our design document.

\subsection{Week 9}
\paragraph{Plan}
This week we will finish the Technology Review and Implementation Plan.
\paragraph{Problems}
This week we need to finish the tech review and implementation plan. We also need to run the existing code for our project.
\paragraph{Progress}
This week I worked on the Technology Review and Implementation plan and submitted it via email.
\paragraph{Summary}
This week we turned in the final draft of our technology review. Due to the holiday, we did not conduct our usual biweekly meeting with our client, but did exchange a few emails.

\subsection{Week 10}
\paragraph{Plan}
This week my partner and I will combine our technology review and send it to our client. We will also do the design document.
\paragraph{Problems}
My partner and I need to combine our technology review documents and fix the errors in them that Kirstin graded us on.
\paragraph{Progress}
 I combined our technology review documents and sent it to my client for approval to be checked off. I started the design document.
\paragraph{Summary}
This week we turned in our design document, exchanged a few coordinating emails with our client on his vision of the design document, and prepared our progress report.

\begin{thebibliography}{9}
\bibitem{1} ”CS461 - CS Senior Capstone”, Eecs.oregonstate.edu, 2017. [Online]. Available: \url{http://eecs.oregonstate.edu/capstone/cs/capstone.cgi?project=320} [Accessed: 22- Nov- 2017]

\bibitem{2} ”USGS.gov — Science for a changing world”, Usgs.gov, 2017. [Online]. Available: \url{https://www.usgs.gov/} [Accessed: 22- Nov- 2017]

\bibitem{3} ]”AVIRIS - Airborne Visible / Infrared Imaging Spectrometer”, Aviris.jpl.nasa.gov, 2017. [Online]. Available:
\url{https://aviris.jpl.nasa.gov/} [Accessed: 22- Nov- 2017].

\bibitem{4} ”XSEDE User Portal — Globus User Guide”, Portal.xsede.org, 2017. [Online]. Available: \url{https://portal.xsede.org/software/globus} [Accessed: 22- Nov- 2017].

\bibitem{5} ”C++”, En.wikipedia.org, 2017. [Online]. Available:
\url{https://en.wikipedia.org/wiki/C} [Accessed: 22- Nov- 2017].

\bibitem{6} "MySQL", En.wikipedia.org, 2017. [Online]. Available: \url{https://en.wikipedia.org/wiki/MySQL} [Accessed: 22- Nov- 2017].

\bibitem{7} "XSEDE, Extreme Science and Engineering Discovery Environment", www.xsede.org, 2017. [Online]. Available: \url{https://www.xsede.org/} [Accessed: 22- Nov- 2017].

\bibitem{8} "AVIRIS, Airborne Visible/Infrared Imaging Spectrometer", aviris.jpl.nasa.gov, 2017. [Online]. Available: \url{https://aviris.jpl.nasa.gov/} [Accessed: 22- Nov- 2017].

\bibitem{9} "AVIRIS-NG, airborne Visible/Infrared Imaging Spectrometer Next Generation", aviris-ng.jpl.nasa.gov, 2017. [Online]. Available: \url{https://aviris-ng.jpl.nasa.gov/}[Accessed: 22- Nov- 2017].

\bibitem{10} "COAL, Coal and Open-pit surface mining impacts on American Lands", capstone-coal.github.io, 2017. [Online]. Available: \url{https://capstone-coal.github.io/} [Accessed: 22- Nov- 2017].

\bibitem{11} "HPC, High Performance Computing", en.wikipedia.org, 2017. [Online]. Available: \url{https://en.wikipedia.org/wiki/Supercomputer} [Accessed: 22- Nov- 2017].

\bibitem{12} "Query Language", en.wikipedia.org, 2017. [Online]. Available: \url{https://en.wikipedia.org/wiki/Query_language} [Accessed: 22- Nov- 2017].

\bibitem{13} "Enterprise Module Service", hpccsystems.com, 2017. [Online]. Available: \url{https://hpccsystems.com/enterprise-services/modules/esp} [Accessed: 22- Nov- 2017].

\bibitem{14} "HPCC High Performance Computing Cluster", hpccsystems.com, 2017. [Online]. Available: \url{https://hpccsystems.com/enterprise-services/modules/esp} [Accessed: 22- Nov- 2017].

\bibitem{15} "OSU Unix HPC Cluster", cosine.oregonstate.edu, 2017. [Online]. Available \url{http://cosine.oregonstate.edu/unix-hpc-cluster} [Accessed: 22- Nov- 2017].

\bibitem{16} "MIT License", en.wikipedia.org, 2017. [Online]. Available: \url{https://en.wikipedia.org/wiki/MIT_License} [Accessed: 29- Nov- 2017].

\bibitem{17} "Apache License Version 2.0", www.apache.org, 2017. [Online]. Available \url{https://www.apache.org/licenses/LICENSE-2.0} [Accessed: 29- Nov- 2017].

\bibitem{18} "GNU General Public License, version 2", www.gnu.org, 2017. [Online]. Available \url{https://www.gnu.org/licenses/old-licenses/gpl-2.0.en.html} [Accessed: 29- Nov- 2017].

\bibitem{19} "Top Open Source Licenses",www.blackducksoftware.com, 2017.[Online]. Available \url{https://www.blackducksoftware.com/top-open-source-licenses} [Accessed: 29- Nov- 2017].
\end{thebibliography}

\end{document}
