\documentclass{article}
\begin{document}
\section{Fall term, Bryce Egley}
\subsection{Week 1}
\paragraph{Plans}
This week I plan to choose my project preferences. 
\paragraph{Problems}
After reviewing the project preferences list there were several projects I didn't think I would have the required knowledge for.
\paragraph{Progress}
I have made my selection on my project preferences.
\paragraph{Summary}
For week 1 I selected my project preferences. My preferences selection was as follows 1. Stock Analytics Website 2. Prison Match 3. RPM (Release Point Maximizer) 4. Campus Events Mobile App 5. Modernize Email List Serve I choose the projects based on what I thought I could contribute to the project as well as  my interest in the particular topic.
\subsection{Week 2}
\paragraph{Plans}
 This week I plan to read into my capstone project once I have been assigned and contact my client.
\paragraph{Problems}
I still don't know what my project is but I should know on Tuesday. 
\paragraph{Progress}
Now we have been assigned to our Project. Our project is Prison Match!
\paragraph{Summary}
This week we were assigned to our projects and we got to meet up with our partners. My partner is Kenny Thompson and my project is Prison Match.
\subsection{Week 3}
\paragraph{Plans}
This week we plan to have a meeting with our client to get more details on our project. 
\paragraph{Problems}
Our project was reassigned because our old client for the Prison Match project didn't want to do a capstone project this year. Also one of our group members dropped the course so now we only have two people. 
\paragraph{Progress}
On Thursday we were able to have a google hangouts meeting with our client. We went over deliverables and the main objectives for our project. 
\paragraph{Summary}
This week we were reassigned from the prison match project to  Coal and Open-Pit surface mining impacts on American Lands Follow-On our client is Lewis McGibbney who works for NASA JPL
\subsection{Week 4}
\paragraph{Plans}
This week my partner and I want to finish and send our problem statement to our client.
\paragraph{Problems}
 In order to finish our problem statement we need to combine the individual problem statements we wrote up into one problem statement which we will send to our client.
\paragraph{Progress}
On Tuesday we put together our problem statements and edited it for errors. We then sent our problem statement to our client in an email to sign off on. We cc'd the professors and the TA in the problem statement email. Our Client, Lewis replied the same night and checked off on our problem statement. 
\paragraph{Summary}
This week went well. My partner and I created our problem statement and submitted it to our client. Our client then checked off on our problem statement. 
\subsection{Week 5}
\paragraph{Plans}
This week my plan is to create the rough draft of requirements document.
\paragraph{Problems}
The problem for this is week is that I'm still not sure which parts of the capstone my client wants in the requirement. We will discuss this with him on Thursday in our weekly meeting. 
\paragraph{Progress}
This week we met with our client and created the rough draft of the requirements document. 
\paragraph{Summary}
This week we needed to create the requirements document. We also had a meeting with our client and TA on Thursday. 
\subsection{Week 6}
\paragraph{Plans}
This week we planned to complete the requirements document and send it to our client for approval. 
\paragraph{Problems}
We need to come up with the requirements for the requirements document due on Friday that we need to send to our client. 
\paragraph{Progress}
As of Wednesday we completed our requirements document and sent it to our client to look over and approve. We already had the rough draft on our github. On Friday our client approved our requirements document. 
\paragraph{Summary}
This week we completed the requirements document on time and submitted it to our client who approved our requirements document after telling us to add a reference section. 
\subsection{Week 7}
\paragraph{Plans}
This week we plan to have a meeting with our client on Thursday.  I also plan to start the technology review assignment.
\paragraph{Problems}
This week we need to meet with our client and discuss some more specifics on the capstone coal project. Specifically what deliverables he wants for the week 9 meeting. 
\paragraph{Progress}
This week I've worked on the technology review and discussed with my partner whether or not we should have a meeting in the Thursday in two weeks since that will be Thanksgiving Day.
\paragraph{Summary}
 This week we met with our client on Thursday. 
\subsection{Week 8}
\paragraph{Plans}
This week we plan to write up the Technology Review rough draft, choose issues to work on in the github for our project COAL-FO and run the existing code.
\paragraph{Problems}
The problem is that we need to create the rough drafts for our technology review and start choosing issues to work on. 
\paragraph{Progress}
This week I ran the tests for our existing code but I'm still working on running the example problems. I also wrote up the technology review document. 
\paragraph{Summary}
 Overall this week went well. I choose an issue to work on github and I was able to run the tests for our project. 
\subsection{Week 9}
\paragraph{Plans}
This week we will finish the Technology Review and Implementation Plan. 
\paragraph{Problems}
This week we need to finish the tech review and implementation plan. We also need to run the existing code for our project. 
\paragraph{Progress}
This week I worked on the Technology Review and Implementation plan and submitted it via email.
\paragraph{Summary}
This week went well now we will combine my partners and I technology reviews and implementation plan and sent that to our client for approval. 
\subsection{Week 10}
\paragraph{Plans}
This week my partner and I will combine our technology review and send it to our client. We will also do the design document. 
\paragraph{Problems}
My partner and I need to combine our technology review documents and fix the errors in them that Kirstin graded us on.
\paragraph{Progress}
I combined our technology review documents and sent it to my client for approval to be checked off. I started the design document. 
\paragraph{Summary}
This week we finished the technology review and started on the design document and progress report. 
\subsection{Week 11}
\paragraph{Plans}
This week we plan to finish the progress report presentation.
\paragraph{Problems}
We need to create presentation over viewing our capstone project. 
\paragraph{Progress}
This week we made the presentation. Each was about ten minutes long and we combined them into one presentation video.
\paragraph{Summary}
This week was a successful way to end our capstone project. We created a presentation over viewing our capstone project. 

\section{Winter term, Bryce Egley}
\subsection{Week 1}
\paragraph{Plans}
This week my partner and I cam back to campus and discussed goals for this quarter on what we would like to accomplish for the capstone. 
\paragraph{Problems}
This quarter we have a lot of work to do on Coal, we must prioritize what we want to complete first.  
\paragraph{Progress}
I managed to get the COAL examples working. I will now go on to fixing issues on the github tracker. 
\paragraph{Summary}
This week was a good start for winter quarter. We need to stay focused and make sure we stay on top of assignments and deadlines. 
\subsection{Week 2}
\paragraph{Plans}
 This week I plan to fix all the problems I encountered when trying to run the examples. 
\paragraph{Problems}
Right now the examples have lots of problems when I try to run them. Example mineral.py, example mining.py, example environment.py all don't seem to work. 
\paragraph{Progress}
This week I fixed most of the problems I found with the examples and pushed this to pycoal. Our client, Lewis, then approved the changes I had made. 
\paragraph{Summary}
Overall this week was very successful. Now we are able to run the pycoal examples. We can see what the hyperspectral images look like after mineral, mining and environmental correlation and we can start fixing other issues on pycoal. 
\subsection{Week 3}
\paragraph{Plans}
This week we plan to get AWS accounts to store hyperspectral imagery on. We also plan to solve one issue on pycoal relating to the Docker Build. 
\paragraph{Problems}
We need to fix the Docker build since right now pycoal does not build on Docker. 
\paragraph{Progress}
This week we fixed the issue with Docker and made changes to the example scripts so now we can run any imagery through them. 
\paragraph{Summary}
This week we fixed problems with the examples as well as fixed issues relating to Docker.
This greatly improves Pycoal since now people who fork/clone/download pycoal will be able to see
That works correctly with Docker. Also that the examples will run correctly. 
\subsection{Week 4}
\paragraph{Plans}
This week my partner and I plan to work on issues on the github issue tracker for pycoal and coal-sds. 
\paragraph{Problems}
For my portion I need to solve two issues on the github issue tracker. One is creating a CLI and which will enable upon install. The other is to improve QGIS install instructions.
\paragraph{Progress}
So far I have solved 2 issues on the github issue tracker for pycoal and managed to get the examples running on my local machine. Over the next two weeks I plan to solve 3 more issues and start work on the export process to XSEDE. 
\paragraph{Summary}
 This week we talked to our client about problems we had
Solveing issues on pycoal. We also decided to prioritize the
XSEDE export process, since the Alpha release is in two weeks in week 6.
\subsection{Week 5}
\paragraph{Plans}
This week I plan to sort out the problems and submit with pull request the three github issues I am currently working on. Which are changing Docker image to use python 3, Create CLI and enable upon install, Improving QGIS.
\paragraph{Problems}
These issues specifically, changing the Docker image to use Python 3 and Creating CLI and enable upon install have a lot of tedious details behind them and involve a lot of reading and planning to get them done. 
\paragraph{Progress}
This week I finished Improving QGIS instructions and getting the Docker image to python 3. I make need to come back to the Docker image with Python 3 issue. I also made a pull request with my commits for changes to the CLI for my client to review. 
\paragraph{Summary}
Overall this week was productive. I tackled 3 key issues on github and in my client meeting we discussed the next 3 issues I should work to solve which are Enabling Ecosis Library, Upgrade Library to Spectral Library Version 7 and Labeling the Docker Image. I hope to finish these before the Alpha Level release then I will have successfully fixed 9 issues on pycoal which is very good progress for the Alpha Level Release. 
\subsection{Week 6}
\paragraph{Plans}
This week on pycoal I plan to finish creating the CLI and then start on Enabling Ecosis Library, Upgrade Library to Spectral Library Version 7 and Labeling the Docker Image. We also have the alpha release where we should get something with AWS or XSEDE working. As well as make the poster and presentation. A very busy week!
\paragraph{Problems}
The CLI is trickier than I thought. But I pretty much finished creating it with help from my client who recommend some slight changes to be made. 
\paragraph{Progress}
This week I created the CLI, the Poster for our capstone and started working on the new issues of Enabling Ecosis Library, upgrading Spectral Library to Version 7 and Labeling the Docker Image. 
\paragraph{Summary}
This week overall was productive. I finished one key issue of creating the cli. Which is needed since we are trying to automate our project. The CLI command line interface will allow users to pass files as arguments to go through mineral, mining and environmental correlations. We also created the poster and Alpha release presentation. 
\subsection{Week 7}
\paragraph{Plans}
This week I plan to solve the issues of Enabling the Ecosis Library and Upgrading our Library to Spectral Version 7. If I can finished those I plan to start working on labeling the docker image and picking up a new issues on the pycoal issue tracker to start working on. 
\paragraph{Problems}
This week to need to figure out the new issues we are going to be working on. I have chosen three issues and Kenny is starting to do more work with AWS and COAL-SDS. 
\paragraph{Progress}
This week I began work on Finding new data to use to run through pycoal. I looked at the AVIRIS-NG site and found some photos of coal, sulfur mines and one coal mine which we could potentially use. 
\paragraph{Summary}
 This week overall was productive. We have taken a new direction from the previous week since we finished the Alpha release. I have began more work on the three issues I mentioned before which should consume my time for the next two or so weeks and Kenny is doing more work getting COAL-SDS to work with AWS(Amazon Web Services)
\subsection{Week 8}
\paragraph{Plans}
This week I plan to download 4 AVIRIS-NG images and run full mineral, mining, environmental correlation at least one of them. This would complete the get more data issue, at least for the time being. I also plan on doing some work to upgrade to spectral version 7.
\paragraph{Problems}
We need to fix problems with GDAL right now we are getting trace errors when running. This isn't a pressing issue though since we can still generate environmental classifications. Fixing this would just make the environmental image more clear. 
\paragraph{Progress}
This week I have finished gathering data by finding a few AVIRIS-NG images of coal and sulfur mines which we can run mineral, mining, environmental correlations on. I wont close the get more issue since this issue could technically be opened forever seeing as though in the future pycoal would be used to run the entire AVIRIS and AVIRIS-NG libraries on and move to XSEDE.
\paragraph{Summary}
Overall this week was productive. I completed more work on getting more data where I found  images of coal and sulfur mines which we can run mineral, mining, and environmental correlations on. Next week I plan to do more work on updating to spectral library version 7 and enabling ecosis library.
\subsection{Week 9}
\paragraph{Plans}
This week I plan to run mineral, mining and environmental correlation on three to four images using pycoal and stage the products in our google drive.  I also plan to fix a bug with gdal pycoal currently has and hopefully finish getting pycoal up to date with spectral version 7. 
\paragraph{Problems}
Right now pycoal runs on spectral version 6 we can classify more minerals if we upgrade to spectral version 7 and generate clearer images. There is also a bug with gdal which effects the environmental correlation images we are currently generating. 
\paragraph{Progress}
This week I fixed the GDAL bug and began work on updating to Spectral Version 7. We had our client meeting where we fixed up some misunderstandings I had with Updating to Spectral Library Version 7.  
\paragraph{Summary}
Overall this week was productive for our capstone project. I was able to fix the bug with GDAL and Kenny was able to get some work done on the COAL-SDS side of things. 
\subsection{Week 10}
\paragraph{Plans}
This week I plan to finish on updating pycoal to use Spectral Library Version 7 and then I plan on getting started on Enabling the Eco-SIS library. If I can complete these we will have fulfilled the portion of our project of updating the algorithms and classification methods of the existing code.
\paragraph{Problems}
This week the problem I need to solve is upgrading pycoal to use Spectral Library Version 7. To do this I need to create a convert function. 
\paragraph{Progress}
This week I made progress on this and finished the update to Spectral Version 7. This week was overall productive even with the barrier of dead week and having a lot of other work in my other courses. 
\paragraph{Summary}
This week was overall productive to the development of pycoal. Spectral Library Version 7 is now being used so the algorithm have been improved which was a goal in our requirements document that we have completed. We can always improve the algorithms more but we have at least met that requirement in one way with this.
\subsection{Week 11}
\paragraph{Plans}
This week I will start work on enabling eco-sis. This will further improve pycoal's classification algorithms. We also plan to make the Beta Release Presentation and Video. 
\paragraph{Problems}
This week the main problem is to create the beta release video. This is due  on Wednesday and will conclude the winter portion of the capstone project. 
\paragraph{Progress}
This week we completed the Beta Release. It is due on Wednesday and we will have completed it by Wednesday.
\paragraph{Summary}
Overall this week and quarter have been very productive.  On the pycoal side of things we got the examples working, we created a cli, we fixed the gdal and qgis installation instructions, we found more data to run mineral, mining, envrionmental correlation on, we updated Docker to python version 3, we got pycoal to use spectral library version 7 to improve the pycoal algorithms and many more things.  This fufilled all the parts of the requirements document hat related to pycoal. Mainly improving upon existing algorithms. I continue to spend time working on pycoal but I should also start giving some of my time COAL-SDS since the pycoal requirements have been met. 

\section{Spring term, Bryce Egley}
\subsection{Week 1}
\paragraph{Plans}
This week we re wrote the requirements document, since we removed the COAL-SDS part of our capstone project. 
\paragraph{Problems}
The COAL-SDS part of our capstone project was removed. This wasn't a big deal for me since I am working on pycoal. 
\paragraph{Progress}
This week I worked more on the Spectral Version 7 issues. And Kenny and I re wrote the requirments document.
\paragraph{Summary}
This week we rewrote the requirements document and I worked on the
USGS Spectral Version 7 issue I have been working on. 
\subsection{Week 2}
\paragraph{Plans}
I have sent out about 15 emails to people working at USGS asking them about Spectral Library Version 7's format. I found one who got me in contact with the correct people. 
\paragraph{Problems}
This week I needed to fix certain issues in my spectral library 7 conversion code that my client wanted. 
\paragraph{Progress}
I made the requested changes to my code and 
Made a pull request to github for my client to review.
\paragraph{Summary}
Overall this week was productive. I have now made contact with the people
At USGS and I made requested changes to my code on github. 
\subsection{Week 3}
\paragraph{Plans}
Get convolved USGS spectral Library 7 envi .hdr and .sli files. 
\paragraph{Problems}
 You can generate convolved spectral library files using prism software or writing conversion code. 
\paragraph{Progress}
I talked to people working at USGS and we got the convolved library files for USGS Spectral Library Version 7.
\paragraph{Summary}
Overall this week was very productive. We got the USGS Spectral Library 
Version convolved files and I tested it on pycoal and it worked correctly. 
\subsection{Week 4}
\paragraph{Plans}
Begin writing tests for Spectral Version7 conversion funciton. 
\paragraph{Problems}
I need to write test cases for the spectral version 7 conversion I created. Once these tests pass this issue will be complete
\paragraph{Progress}
I began writing these tests I pushed an initial version to github. Some tests failed and there are changes I need to make. 
\paragraph{Summary}
Overall this week was productive since we finished the spectral version 7 issue
Now we just need to have test cases to make sure it works correctly. 
\subsection{Week 5}
\paragraph{Plans}
Finish tests for USGS Spectral Library 7 conversion method. Update website to reflect this. 
\paragraph{Problems}
The USGS Spectral 7 update issue is complete now we just need to update the website and fix the broken travis-ci.
\paragraph{Progress}
This week was very productive. We are nearing the end of the capstone. And this issue being solved will be a major addition to pycoal.
\paragraph{Summary}
Overall this week was very productive. We managed to updated to spectral 7, Now pycoal will be able to identify more spectra and produce more accurate correlation images.
\subsection{Week 6}
\paragraph{Plans}
Enhance aster conversion function to work with EcoStress Library. Discuss with my client if convolving ECOSIS Spectral Library is really possible in current format.
\paragraph{Problems}
I wish to apply my same method applied to Spectral Version 7 to the EcoStress Spectral Library and possibly Ecosis. By doing this we will allow pycoal to work with more spectral libraries. 
\paragraph{Progress}
This week was the last week before the code freeze. Overall I have made significant modicifications to pycoal and met the requirements I had started out on.
\paragraph{Summary}
This capstone project was very successful. I certainly learned a lot about software development. And I look forward to applying what I've learned through this project to my future career.
\subsection{Week 7}
\paragraph{Plans}
This week the plan is to prepare for the engineering expo. Which is on Friday
\paragraph{Problems}
We have finished all the code we need to. I may keep running mineral correlation images on my Desktop computer so my client will have more samples to show for this project, if he wishes to continue it in the future.
\paragraph{Progress}
We have made our expo poster, completed all documentation and all coding requirements we set out with. All we have to do now is present our project at Expo!
\paragraph{Summary}
Overall this has been a very productive capstone project and now we will present our year long project at the engineering expo!
\subsection{Week 8}
\paragraph{Plans}
This week just have to do documentation following expo making sure that we meet all the requirements for the class.
\paragraph{Problems}
We will touch base with our client and our making sure we meet all documentation requirements. 
\paragraph{Progress}
Overall this has been a productive week and we have completed everything we need to for the engineering capstone project. 
\paragraph{Summary}
This capstone project has been very rewarding it terms of getting experience of developing software and working with a client. I will definitely be able to use these skills in the future. 
\subsection{Week 9}
\paragraph{Plans}
This week we worked on documentation for the final report. 
\paragraph{Problems}
All capstone work besides the final report is completed.
\paragraph{Progress}
We worked on the final [presentation and report for the capstone project.
\paragraph{Summary}
Overall this week was productive as we close down the capstone project.
\subsection{Week 10}
\paragraph{Plans}
This week we need to make the final presentation for the capstone project.
\paragraph{Problems}
The final presentation needs to have a demo of how to use all our projects features.
\paragraph{Progress}
This week we finished the final presentation and submitted it to canvas.
\paragraph{Summary}
Our final presentation is complete and now all that is left is the final report.
\subsection{Week 11}
\paragraph{Plans}
This week we need to make the final report.
\paragraph{Problems}
To finish our capstone project we need to write up the final report.
\paragraph{Progress}
This week we wrote the final report and finished the capstone project!
\paragraph{Summary}
It has been a great capstone project and a great year! Thank you for giving me this experience and I look forward to applying what I have learned through this project in my future career!

\end{document}
