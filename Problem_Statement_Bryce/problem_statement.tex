% ===============================================
% CS 461: Problem Statement         Fall 2017
% hw_revised.tex
% Template for revised homework submission
% ===============================================
%         READ THE FOLLOWING CAREFULLY!!!
% ===============================================
% When you produce a PDF version of this document
% to turn in, change the filename to hwX-name.pdf
% replacing X with the homework assignment number
% and name with your last name.
% ===============================================

% -----------------------------------------------
% The preamble that follows can be ignored. Go on
% down to the section that says "START HERE"
% -----------------------------------------------

\documentclass{article}

\begin{document}

% ------------------------------------------ %
%                 START HERE                 %
% ------------------------------------------ %

\title{Problem Statement: Coal and Open-pit surface mining impacts on American
Lands Follow-On (COAL-FO)}
\author{Bryce Egley\\CS 461: Senior Capstone} % Replace "Author's Name" with

\maketitle

% -----------------------------------------------------
% The following two environments (theorem, proof) are
% where you will enter the statement and proof of your
% first problem for this assignment.
%
% In the theorem environment, you can replace the word
% "theorem" in the \begin and \end commands with
% "exercise", "problem", "lemma", etc., depending on
% what you are submitting. Replace the "x.yz" with the
% appropriate number for your problem.
%
% If your problem does not involve a formal proof, you
% can change the word "proof" in the \begin and \end
% commands with "solution".
% -----------------------------------------------------

% ---------------------------------------------------------
% Abstract 100-150 Words
%----------------------------------------------------------
\section*{Abstract}
Bryce Egley
\newline
CS 461 Problem Statement: Coal and Open-pit surface mining
\newline
Abstract
\newline
It is important for us to build models to monitor the effects of coal and open
pit surface mining on American Lands. To do this our project will have four
primary objectives we intend to complete.
We want to improve on the existing method of COAL imagery processing algorithms
which identify mining activity in the U.S.
We want to improve the accuracy of finding correlations between location of
mining and related activities and their effects on those nearby lands. Such as
comparing similar areas land in which certain practices and methods of mining
have been in use.
We want to come up with a baseline suite of reporting metrics to appropriately
rank and document the changes within land as well as streams or other natural
barriers and landmasses nearby.
We want to extend this project 'COAL' so it can be have reusable components
that can be used cloud platforms for other projects or researchers who wish to
apply this program to their own field.

\newpage
% ---------------------------------------------------------
% Definition and Description
%----------------------------------------------------------
\section*{Definition and Description}
This project will create new and improve upon methods to classify, characterize
and quantify impacts of mining and other destructive surface mining activities
across the United States. The COAL project recently won an award and funding to
go on XSEDE a single virtual system where individuals will be able to share
computing resources. We will make this project work on XSEDE so other
individuals will be able to use the program and data that comes from the COAL
project. Putting the COAL project on XSEDE will enable us to further improve
the algorithms, execution runtime performance and geospatial output results.
The original COAL project focused on gathering data from remote sensing devices,
providing a suite of algorithms for classifying land cover, identifying mines
and other graphical features, and correlating them with environmental data sets
as described on the coal capstone github page. We will also spend time on this
project reviewing and understanding the previous code that was done by the
group before us since this is a follow on project. Along with updating the blog
for the COAL follow on project in hopes that future groups or people trying to
incorporate COAL into their own project will be able to understand our goals
and how our code functions. This is the fourth capstone project and hundreds of
people have forked the code for this project. So it is very important that we
document our work so that people using our code and data collected in the
future will be able to easily understand what we did. It is also very important
to track and monitor the environmental changes and effects coal has on
surrounding lands since Coal is one of Americas main sources of energy. We of
course want to make sure we are extracting goal and using it safely for the
sake of the environment and for people who live near these areas. Satellite
imaging using remote sensing devices allows us to gather much more data than a
human eye would be able to. Satellite imaging can gives us a picture of an
entire area of land and see things such as methane emissions which a human eye
would not be able to see. The NASA Jet Propulsion Lab has deployed drones and
satellites to gather data which we will be able to use for this project.

% ---------------------------------------------------------
% Proposed Solution
%----------------------------------------------------------
\section*{Proposed Solution}
In order to solve the problem addressed so far we wish to put the COAL-FO (COAL
follow on) project on XSEDE. XSEDE only allows for 2 terabytes so we will come
up with a method to loop through the data so that users can pull up any of the
images on XSEDE. The main priority for our capstone project is making a
searchable port for the COAL project. We also plan on having meetings every two
weeks where we will turn in and discuss past week deliverables and plan future
deliverables. This is where we will create a schedule of how to complete the
project. As already discussed we will also plan on improving the existing
COAL architecture and systems. We also hope to improve the imagery processing
algorithms currently in use. Then we will hope to create a new baseline suite
to rank changes to areas of land over time to determine if the changes in one
particular area our significant or not.

% ---------------------------------------------------------
% Performance Metrics
%----------------------------------------------------------
\section*{Performance Metrics}
The performance metrics for this capstone project will be if we sufficiently
meet the objectives as discussed in the abstract and the project description.
Our main priority according to our client is to create a searchable port for
the project. Once we complete this we will move on to the other objectives such
as putting the project on XSEDE and improving accuracy and time of the current
algorithms that are in use. We will also have performance metrics from week to
week based on the meeting we have with our client and the deliverable we choose
to set every other week. If we can complete those and the high priority
objectives then this project will be a success. We don't necessarily need to
complete all the objectives for this project to be successful as long as the
high priority ones like the searchable port and putting the project on XSEDE
are done or mostly finished. The objectives while important should only be
worked on once we have completed work on the high priority objective discussed.
These lower priority objectives would include improving existing algorithms
and improving the accuracy of finding correlations between land changes in
different images. Overall this project looks challenging and it seems like a
lot of work but if we focus on one objective at a time and use the resources
available to us I believe we can accomplish several of the high priority
objectives. We also hope that the work we do on this project will make it more
accessible and usable for other individuals working on their own projects and
research. We look forward to working with the NASA Jet Propulsion Lab and the
members of COAL project from last year.

% -----------------------------------------------
% Ignore everything that appears below this.
% -----------------------------------------------

\end{document}
